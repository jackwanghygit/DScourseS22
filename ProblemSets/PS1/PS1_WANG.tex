\documentclass{article}

% Language setting
% Replace `english' with e.g. `spanish' to change the document language
\usepackage[english]{babel}

% Set page size and margins
% Replace `letterpaper' with`a4paper' for UK/EU standard size
\usepackage[letterpaper,top=2cm,bottom=2cm,left=3cm,right=3cm,marginparwidth=1.75cm]{geometry}

% Useful packages
\usepackage{amsmath}
\usepackage{graphicx}
\usepackage[colorlinks=true, allcolors=blue]{hyperref}

\title{PS1}
\author{Jack Wang}

\begin{document}
\maketitle

\section{Introduction}

Currently I am a second year PhD student in accounting and my research focuses on financial archival research with a special interest in disclosure issues. I believe my research will benefit a lot from the knowledge in data science, because currently publication in top journals becomes more and more competitive and proprietary data would be an important way to make my research stand out. Techniques in data science will enable me to generate some proprietary data related to disclosures most of which are in textual format.

I do not have a specific idea about the project for this class at this stage. I am trying to develop some ideas about disclosure that will involve qualitative information analysis. Research using tweet data is an emerging area in accounting. Maybe I can try this direction.

I believe it might be unrealistic to handle skills in data science by taking a class. Hopefully, I can learn what tools are available in data science that can be applied to my future research and where I can get access to information about these tools if I want to proceed to a more advanced level. It will take me three more years to graduate, but the career path for accounting PhD students is clear--being a professor, hopefully in a research school.


\section{Equation}

\[a^2+b^2 = c^2\]

\end{document}